\chapter{Development}

\section{Typical Component Interaction Within A \node{}}

\begin{figure}[htbp]                                                        
  \resizebox{.8\textwidth}{!}{ \includegraphics{figures/sequence.eps}}    
  \caption{Typical Interactions Within a \node{}}                                         
  \label{fig:sequence}
\end{figure}  

\fig{sequence} shows the interaction between \coven{} components.
This diagram was simplified for legibility. In reality, many of the
components have their own asynchronous queueing mechanism, effectively
connecting 2 threads through a synchronized queue. This is only explicitly
shown in the \compproto{}OutPlugin\slash{}\compproto{}Conn interaction,
but will also be implemented in e.g. the ZeroMQ implementation of the
\cauldron{} and many read and write \witches{}

We show one (unspecified) write \witch{}, one specific read \witch{} (\diffproto{}
server backend), and no read/write \witches{}.

From top to bottom:

\begin{itemize}

\item The \coven{} has launched one write and one read \witch{}, subscribed
to the \cauldron{} for its read \witch{}, and opened a PUSH type channel
for each read \witch{}. The latter being stateless, it is not shown.

\item The input \witch{} has values to contribute, hands them to the
\coven{} by writing them to stdout as \rawproto{} shuttles, whenever it
has them (which may be frequently).

\item The coven writes the values gathered (possible consolidated)
to the \cauldron{}'s PUSH channel.

\item The \cauldron{} copies the values received by any input \witches{},
(possibly consolidated) to all subscribers.

\item The coven{} writes the values received from the \cauldron{}, to
the stdin pipes of all output \witches{} (possibly filtered for a particular
subset of the namespace)

\item The \diffproto{}OutputPlugin updates its cache with the values
received.  While no clients are connected to its front-end listener,
that is all it does.

\item When an HTTPS connection is accepted from the Internet, the web
server makes a back-end connection to the \diffproto{}OutputPlugin's
listener.

\hyphenation{con-nection}
\hyphenation{in-stan-ti-ates}
\item for each new connection, the \diffproto{}Plugin instantiates
a \diffproto{}Connection and hands over the connected socket, which
handles the connection from reading the HTTP request and until the
connection closes.

\item the \diffproto{}Connection interprets the HTTP GET, and URL
information, possibly sets filters to limit the output, to set variants
in its format, etc..

\item the \diffproto{}Connection requests a cache lock from the
\diffproto{}Plugin, and requests the entire cache contents (possibly
filtered), which it sends to the webserver, and releases the lock.

\item While 1..n clients remain connected to the \diffproto{}Plugin
via its \diffproto{}Connection instances, any cache updates will also
be journalled, and the differences copied to all outgoing queues, where
they will be sent over the existing connection at the tempo attainable
by the specific client. This is the response to incoming data for as
long as there are clients connected.

\item to terminate, a client closes its TLS connection, which will
cause the corresponding \diffproto{}Connection to unregister itself
and terminate.

\end{itemize}

\section{ExtreMon Viewer Runtime}

\begin{figure}[htbp]
	\resizebox{.8\textwidth}{!}{ \includegraphics{figures/classes.eps}}
	\caption{Runtime Class Diagram}
	\label{fig:classes}
\end{figure}

The in-memory representation of the relationships established between the
\rawproto{} namespace and the elements in the SVG DOM Tree maintained
by the SVGCanvas from the Apache Batik library is represented in
\fig{classes}. The central class is an X3Panel which contains
the Batik SVG document, doctored as described above, a dictionary storing
the variables established by x3mon:define attributes, above, a queue
of Sets of Alteration objects: One Alteration encapsulates exactly one
.. alteration to either an attribute or the text content of exactly one
SVG element in the DOM, and a dictionary mapping certain \rawproto{}
identities to sets of SetAction subclasses that they should cause. Every
x3mon:map instance in the original SVG document instantiates one SetAction
subclass, according to its type: ``tset'' instantiates a TextSetAction,
``nset'' a NumericSetAction, and so forth.

\subsection*{Mapping}

\begin{figure}[htbp]
	\resizebox{.8\textwidth}{!}{
	\includegraphics{figures/mappings.eps}} \caption{From \rawproto{} records to SVG DOM elements}
	\label{fig:mappings}
\end{figure}

\fig{mappings} illustrates how incoming record are converted
into SVG alterations at runtime: on the left is an associative array whose
keys are the full \rawproto{} labels (truncated in front to save space
in the drawing), mapping to values that are lists of SetAction subclass
instances. An incoming label is looked up in the associative array,
and this yields a list of SetActions to be executed in response. Every
SetAction maintains a reference to the SVG element to act upon, the
attribute of that SVG element to act upon (or null indicating that the
text content of the element is the target instead), the format string to
format the data first, and any modifiers, according to its type. However,
this does not suffice to alter the SVG DOM: The Batik library does not
allow its DOM representation to be altered from just any thread, only
from its own update queue, and the updates can only be passed to it as
instances of the Java Runnable interface.

\subsection*{Batching Updates}

\begin{figure}[htbp]
	\centering
	\resizebox{.5\textwidth}{!}{\includegraphics{figures/viewersequence1.eps}}
	\caption{X3Mon Viewer Calling Sequence Phase 1}
	\label{fig:viewersequence1}
\end{figure}

While the SVG DOM is exposed, and may be directly manipulated, this
leads to undefined behaviour. Because the runtime cost of each Runnable
object is relatively high vis-a-vis the number of alterations per second
we expect, we'll present Batik's update queue with a type of Runnable
that will set as many objects as once as is optimal. Our Alternator
Class (not shown) is merely a Runnable containing a Set of Alterations,
that, in its run() method calls each Alteration's alter() method. Our
X3Panel class, then, receiving records, looks up corresponding SetAction
subclasses, but these SetActions merely call X3Panel's enqueueAlteration
to add the required alterations to its alteration queue. This is shown
in \fig{viewersequence1}

\begin{figure}[htbp]
	\centering
	\resizebox{.5\textwidth}{!}{\includegraphics{figures/viewersequence2.eps}}
	\caption{X3Mon Viewer Calling Sequence Phase 2} \label{fig:viewersequence2}
\end{figure}

In a separate Thread, shown in \fig{viewersequence2}, X3Panel
unqueues these Alterations, gathers them into Alternator instances
and presents them to Batik's update queue.  This intermediary stage is
where it will be possible to tune the number of updates per Alternator,
according to how the Batik library has tuned updates, which it does
according to the capabilities of the underlying platform.

\subsection*{Updating The DOM From Batik's updateQueue}

\begin{figure}[htbp]
	\centering
	\resizebox{.5\textwidth}{!}{\includegraphics{figures/viewersequence3.eps}}
	\caption{X3Mon Viewer Calling Sequence Phase 3}
	\label{fig:viewersequence3}
\end{figure}

Batik will then, in its own good time, apply them to the SVG DOM,
eventually rendering them visible. \fig{viewersequence3} shows
how Batik's updatequeue dequeues X3Mon's Alternator instances and runs
them, causing various SVG DOM elements to be modified, inside Batik's
update thread, and at a time appropriate to fluent visual updates (as
managed by Batik).

\subsection*{The Result: Updated SVG DOM}

\begin{figure}[htbp]
	\resizebox{.8\textwidth}{!}{
	\includegraphics{figures/examplerecords.eps}}
	\caption{Example Incoming \rawproto{} Records}
	\label{fig:examplerecords}
\end{figure}

For the example incoming \rawproto{} records in
\fig{examplerecords}, eventually, our example SVG document from
\ref{fig:svgsubst} should end up being transformed and reach the state
as shown in \fig{svgaltered}: The appropriate elements altered
according to format specified.

\begin{figure}[htbp]
	\resizebox{.8\textwidth}{!}{
	\includegraphics{figures/svgaltered.eps}}
	\caption{Example SVG DOM as altered by incoming records}
	\label{fig:svgaltered}
\end{figure}

Finally, \fig{visualexample} shows what the altered SVG may
look like, represented on the visual display (for cpu.0)

\begin{figure}[htbp]
	\centering \resizebox{.5\textwidth}{!}{
	\includegraphics{figures/svgaltered_bars.eps}}
	\caption{A Visual Example According To The Altered SVG DOM}
	\label{fig:visualexample}
\end{figure}

\lstset{ numbers=none,numbersep=0pt,tabsize=2,breaklines=true,resetmargins=true,linewidth=\textwidth }

\section{Core Classes}

The core operation at the heart of many other classes is the assembly and
timely transmission of \rawproto{} shuttles (see \ref{sec:protocol}).
The Loom classes encapsulate the desired behaviours of assembling and
transmitting shuttles of a maximum size, and containing records of
a maximum age. Thus, the calling class may blindly use the Loom's put
methods to add individual records, and the Loom instance will accumulate
these until either there are almost too many of them (max shuttle size)
or too much time has elapsed (max shuttle age), at which point it will
call its assigned launcher instance, delegating the actual transmission of
the shuttle data, since shuttles may be transmitted in various ways. For
example, the chalice\_out\_http plugin will write shuttles to an open
TCP connection inside an existing HTTP session, the \coven{} will write
shuttles to unnamed pipes for its plugins, and multicast them over
UDP packets, the dump tool will write shuttles to stdout. For example:
The CauldronSender class is a Loom subclass, and merely provides for
UDP multicast transport, inheriting everything else from Loom.

\subsection{CauldronServer}

CauldronServer is the base class for creating servers that handle
distributing shuttles to multiple clients. It handles distribution and
allows each client their own private queue, so that slow clients do not
slow down operations for fast clients.

\subsection{ChaliceServer}

ChaliceServer is a specialization of CauldronServer that implements
the differential protocol (see \ref{sec:protocol}), by sending
each new connecting client a full set of known values, and only changes
afterwards.

The Server base classes have only been implemented in Python.

\subsection{DirManager}

DirManager is a utility class that encapsulated the inotify functionality
if the Linux Kernel, as exposed by the pyinotify module, expands it
with MD5 checksums, and exposes an interface that calls methods from
its ``watcher'' delegate with files have been added to, removed from,
or changed inside the directory it is monitoring. The \coven{} class
uses a DirManager to watch its plugins directory, allowing plugins to
activate automatically when added to the plugins directory, shut down
when removed, shut down and restarted when modified.

\subsection{\coven{}}

The \coven{} class is at the heart of ExtreMon. A Coven instance monitors
a plugin directory (using a DirManager) and activates the files there
as ExtreMon plugins, with a hot-plug strategy. Files added (or modified)
are scanned for ExtreMon plugin headers. If such are found, the file is
associated with a Plugin instance, which will attempt to execute the
file, with the configuration headers in its execution environment,
and with its stdin\slash{}stdout file descriptors redirected to
Plugin2Cauldron\slash{}Cauldron2Plugin instances, according to its
interpretation of the presence of input and\slash{}or output filters:
Plugins with input filters (\#x3.in=) get a Cauldron2Plugin on their stdin
descriptor, Plugins with output filters (\#x3.out) get a Plugin2Cauldron
instance on their stdout. The Plugin instance blocks reading from the
plugin's stderr, passing any lines read to the \coven{} instance for
logging purposes: Anything a plugin writes to stderr ends up in the
\coven{}'s log, preceded by the plugin's name. Finally, the \coven{}
class maintains some statistics about its running plugins, and gathers
statistics from Cauldron2Plugin and Plugin2Cauldron instances running,
to contribute using its own CauldronSender instance. The counting
functionality is encapsulated in the Countable class.

\subsubsection{Cauldron2Plugin}

Cauldron2Plugin instances set up a CauldronReceiver read every shuttle
that boils, and assemble these into new shuttles containing only records
matching the plugin's input filter \emph{but not its output filter, if it
has one}.  These are then written to the plugin's stdin. The limitation
on the output filter is to avoid endless loops: If the plugin's logic
were to respond to a record matching its input filter with another record
that would, it would endlessly repeat that record. Hence, input plugins
that are also output plugins are not allowed to see their own records.

\subsubsection{Plugin2Cauldron}

Plugin2Cauldron instances set up a CauldronSender read every shuttle
from the plugin's stdout, passes any that match the plugin's output
filter to the CauldronSender for contribution to the boiling records.

\section{Plugin Facilitator Frameworks}
\label{sec:pluginfacilitators}

While the protocol used in ExtreMon plugins was intentionally kept
extremely straightforward: (for input: read lines from stdin, split them
around an equals sign, part zero is the label and part one the value,
ignore empty lines (end of shuttle), to output, write label=value pairs
to stdout, an empty line if you want your records to be transmitted
ASAP), this does not mean that it is as trivial do write a CPU-efficient
plugin when faced with high volumes of records. Also, there are some
tasks that many plugins will have in common. Many plugins will want
to match parts of their input filter, and many, for example aggregator
plugins operating on multiple records at once, will want to maintain a
cache of certain values.

\subsection{X3In}

By subclassing X3In, a plugin written in the
Python language will have its overridden ``receive'' method called for
incoming shuttles matching its input filter. Each shuttle is available
as a dict() argument.  In the constructor, the subclass may choose to
enable the caching and\slash{}or capture facility, where caching means
that the X3In superclass will keep a dict() of the last values received,
and capture means the X3In superclass will call the receive method with
an extra argument: a dictionary of all the named regular expression
group matches present in the input filter.

\subsection{X3Out}

The XOut superclass gives a plugin written in the Python language the
freedom to call its put() method whenever it has values to contribute, and
to have these sent as shuttles on its stdout, in a timely and efficient
manner, even if there are a great many of them, or in the other extreme,
when there are very few.

X3Out also has a Java implementation.

\subsection{X3IO}

X3IO is a convenience class that makes its subclass both an X3In and
X3Out.

\subsection{X3Conf and X3Log}

All X3In, X3Out, and X3IO instances are also X3Conf and X3Log instances,
giving their subclasses access to the plugin configuration data
(extracted from the environment variables set by the \coven{}), and
exposing the log() method by which a plugin may write log entries into
the \coven{}'s log.

\section{Network Client Facilitators}
\subsection{X3Client}

Although writing a \rawproto{} or \diffproto{} client was intentionally
made as simple as possible, and the simplest cast comes down to making an
http connection, and parsing individual lines after the HTTP headers for
label-value pairs, even this simple task may get more complex once https,
mutual authentication, or multiple \diffproto{} sources are to be read
at once, and when reading large volumes or records.  The \diffproto{}
client classes have been created to greatly simplify such tasks.

To receive shuttles belonging to the part of the namespace it is
interested in, an application would instantiate an X3Client add as
many X3Sources as it has \diffproto{} servers to connect to for that
namespace (for example, the chalice\_out\_http plugin)
and add itself as an X3Client listener.  The X3Client will then call
the application's clientData method, passing it all the records that
have changed from the previous call. If more than one X3Source have
been added, the X3Client will merge the results and only pass the most
recently received values. In this way, the application will continue
to receive records, without any interruption, even if one redundant
X3Source's connection is lost, for whatever reason.


To be able to receive X3Measure records , the application should implement
the X3ClientListener interface.

An application interested in the state of X3Source connections might
also implement the X3SourceListener interface , and add itself to
the individual X3Sources, to obtain notification of connections and
disconnections.  In the same way, an application might choose to use
only a single X3Source, and receive that source's records directly,
without the mediation of an X3Client. An X3Source has a socketFactory
setter, allowing alternate implementation to be injected, for example,
to support Mutual SSL with tokens, as we do in the prototype.

\subsection{X3Drain}

The counterpart of an X3Source is an X3Drain , which allows an
application to contribute records to a \diffproto{} server, such as
the chalice\_in\_http plugin described above. An X3Drain is a Loom, and
will therefore disconnect the application's lifecycle from the sending
of shuttles: The application is free to contribute individual records,
the X3Drain will transmit them at most max\_shuttle\_age later, in its own
thread, by HTTP POST. Because of the semantics of the HttpsURLConnection
class used, HTTP connections that are regularly used will actually make
use of one TCP connection, using the HTTP keepalive mechanism. This means
that as long as there are records available every few seconds, shuttles
will arrive at the server with very little delay, as the connection is
already open, and any SSL handshake has been performed. This is important
especially when using Mutual SSL, with smart card tokens, where the
SSL handshake may take several seconds. An X3Drain has a socketFactory
setter, allowing alternate implementation to be injected, for example,
to support Mutual SSL with tokens, as we do in the prototype.

\section{ExtreMon Viewer}

The Visualisation Console (``Console'') is a Java application, deployed
by Java WebStart, that will open a live schematic view for any Java-
enabled client.

\subsection{X3Panel}

The main classes of the ``X3 Console'' are the X3Panel which contains
an Apache Batik JSVGCanvas representing an SVGDocument into which an
SVG schematic with a supplementary namespace is loaded.
The X3Panel parses the SVG tags inside the schematic
and builds an internal structures that allows it to rapidly translate
incoming \diffproto{} records into alterations to the SVG DOM maintained
by the JSVGCanvas.


Since our chalice\_out\_http plugin plugin will interpret the HTTP GET
path it is called with as a regular expression, and use that to select
which subset of the namespace to transmit, here is an opportunity to
fine-tune bandwidth usage: The X3Panel, after parsing the SVG tags
linking the schematic to the ExtreMon namespace, has a list of any
and all \diffproto{} labels it is capable of responding to. Because
it makes no sense to subscribe to a subset it could not respond to,
the Subscription class exist to build a specialized regular expression
based on the labels an X3Panel can respond to, to allow an X3 Console
to receive only relevant records, and gain bandwidth and CPU cycles.

\subsection{Respondable}

Some SVG elements in the schematic will represent states, and may
therefore become eligible to connect operator feedback to those
states. Elements thus selected are stored by the X3Panel as instances
of Respondable

\subsection{Actions}

The Action classes are the runtime representation of the action tags
present in the supplementary SVG namespace.

\subsection{Alteration and Alternator}

Since the SVGCanvas instance exposes the underlying DOM, but allows no
alterations outside a specialised internal Thread, the Action classes can
never directly alter the DOM (technically, they could, but this leads to
undetermined behaviour and ultimately, a crash), instead, they convert
their actions into Alterations, which contain the low-level information on
what to alter on which element. The Canvas' UpdateManager may be passed
a Runnable, that is executed inside the Canvas's update thread and this
runnable is the only way to reliably alter the SVG.  To avoid having to
pass a great number of Alterations, all Runnables, which would be very
inefficient, a number of Alterations are gathered inside an Alternator and
this is passed to the update manager instead.

\subsection{Supporting Classes}

A few supporting utility classes and interfaces follow. SVGUtils
encapsulates convenience methods to making
working with the SVG DOM more readable and less verbose. ResponderListener
is an interface which allows an application to listen for operator actions
on Respondables. This is how an X3Console detects an operator responding
to and commenting on a visual state. X3PanelListener is an interface
which allows its implementor to know when an X3Panel is ready loading and
parsing the schematic. This is how an X3Console knows that the schematic
is ready, and it waits for this event before connecting to any X3Sources.

\subsection{User Interaction}

While arguably, user interaction could have been done entirely in SVG,
development time dictated that we use a more classical approach for
those few user interactions to be handled at this stage.

A CredentialsDialog is presented when an X3Source encounters a HTTP
authentication request.  It requests the username and password. This
could happen, for example, if a front-end webserver were configured for
HTTP Simple Authentication.

An X3UserAgent  is an implementation of an SVGUserAgentAdapter, which
is the way the Batik classes communicate problems. Our implementation
simply logs anything it receives. It was essential to implement this and
replace the default handler since the latter creates an error dialog
for each problem, potentially overwhelming the user with dialogs for
even a small misconfiguration.

A ResponderCommentDialog uses a JOptionPane to present operators with
a simple dialog into which to type their comments when responding to
a state.

\subsection{X3Console}

Finally, the X3Console class itself, which brings all this together.
An X3Console instantiates an X3Panel inside an undecorated JFrame of
exactly Full HD Size, sets a default authenticator that will request
credentials from the user with a CredentialsDialog and asks the X3Panel to
load the SVG schematic.

When the SVG is loaded (and interpreted for actions), the X3Panel calls
panelReady() and the X3Console creates an X3Client and subscribes to it,
creates an X3Source, retrieving the subscription regular expression
that the Subscription class inside the X3Panel has prepared while
interpreting the SVG.

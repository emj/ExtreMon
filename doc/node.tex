\chapter{\node{} Structure}
\label{sec:node}

\section{General Structure}
\label{sec:nodegeneral}
                                                                          
\begin{figure}[ht]                                                        
    \centering \includegraphics{figures/x3mon_basic.eps}                   
    \caption{\node{} Components} \label{fig:x3monbasic}                    
\end{figure}

The high-level structure of ExtreMon in an Internet context is shown
in \fig{x3monbasic}: A centralised \node{} passively receives
internal measures from various hosts using CNP (collectd's UDP-based
network protocol), actively performs various functional tests on
these and possibly other hosts and services, and serves the results
and derived results to a number of clients, for example the ExtreMon
Display client, over the network, using the DTMP protocol. The dots in 
\fig{x3monbasic} indicate that normally, there will be many
hosts monitored, at least two ExtreMon \node{}s, and various clients.

\section{Inside a \node{}}
\label{sec:nodeoverview}

\begin{figure}[ht]
    \centering \includegraphics{figures/x3mon_node.eps}
    \caption{\node{} Components} \label{fig:x3monnode}
\end{figure}

The structure of a \node{} is shown in \fig{x3monnode}:
The central component of a \node{} is the ``\cauldron'' around
which the other components gather. The \cauldron{} is an N..M 
publish-subscribe mechanism: Any number of writers may present 
it with shuttles (see \ref{sec:protocol}), it will 
duplicate them and send a copy of any received shuttles to all
subscribers. To uncouple the other components from whichever
mechanism is used to achieve this, the \coven{} component
connects to the \cauldron{} and abstracts away its pub-sub
functionality behind a set of simple roles, to be implemented
by a set of \witches{}. \witches{} fit into one of 3 types:

\section{Write \witches{}}
\label{sec:write\witches{}}

Write \witches{} acquire measures from some arbitrary source
and write these to the \coven{} as \rawproto{} shuttles within a 
particular subset of the namespace. The \coven{} feeds these
measures into the \cauldron{} using its publish mechanism. 
In \fig{x3monnode}, a,b, and e represent write \witches{}:
a reads \collectdproto{} packets from a collectd instance, b probes external 
web services and measures their response times, e receives 
operator feedback.

\section{Read \witches{}}
\label{sec:read\witches{}}

Read \witches{} receive shuttles containing a particular subset
of the namespace, from the \coven{}, who feeds them by maintaining
subscriptions to the \cauldron{}'s subscribe mechanism, for them.
In \fig{x3monnode}, d and f represent read \witches{}:
d has a local \diffproto{} listener to which the webserver directs \diffproto{}
requests from the Internet: All external \diffproto{} clients will end
up being served by this important \witch{}. f is a specialised
read \witch{} that feeds measures into third-party graphing engine.

\section{Read\slash{}Write \witches{}}
\label{sec:rw\witches{}}

Read\slash{}Write \witches{} are Read and Write \witches{} combined:
They both read values from the \coven{}, and contribute other values.
While other cases are certainly possible, Read/Write \witches{}
will most often supply aggregation functionalities, reading measured
values and contributing new values derived from these.
In \fig{x3monnode}, c represents a read/write \witch{},
clearly an aggregator, for its lack of external connections.  For example,
c could be taking absolute values, say, free bytes on disk partitions,
or free bytes in memory, etc.. Combining them with other measures to
produce the same information in percentage form, which is easier to
formulate general thresholds against.  Another read/write \witch{} could
then be taking these percentages, applying threshold conditions to them
and contributing alert states, or trending them to predict alert states.

Current \node{}s rely heavily upon a local instance of the collectd agent
to receive and aggregate measures from various collectd agents deployed
on the hosts to be monitored, and an instance of a webserver, to supply
SSL termination, authentication, authorisation, and integration with
other services, like graphing engines. The \coven{}, and any of the
current DMTP \witches{} have no form of security whatsoever: while they
may read metadata supplied by the webserver, the veracity of this is
entirely left to the latter. This was done intentionally.

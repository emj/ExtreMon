\chapter{Protocols}
\section{The \rawproto{} Shuttle}
\label{sec:protocol}

\begin{figure}[ht]
	\resizebox{1.0\hsize}{!}{ \includegraphics{figures/protocol.eps}}
	\caption{The \rawproto{} Shuttle} \label{fig:rawproto}
\end{figure}

An \rawprotoacronym{} shuttle is, intentionally, text-based, human-readable
and simple. \fig{rawproto} expresses its constituent parts,
explained below:

\begin{itemize} 

\item records are sent as label-value pairs, with
the ASCII ``='' (equals) character (hexadecimal value 3d) separating
label and value, and ASCII linefeed (hexadecimal value 0a) separating
records (but being considered part of the record) 

\item where records
are logically grouped, in terms if having one or more common elements
such a group is called a shuttle. Shuttles are separated by an empty
line, that is, two record separators (and this is considered part of
the shuttle) 

\item a label consists of label-items separated by ASCII
``.'' (full stop) character (hexadecimal value 2e). Label-items either
represent an information hierarchy, general-to-specific from left
to right, or a collection of labels whose position is not relevant.
Only the first situation will be considered here.

\item a value consists of a decimal number, with arbitrary precision,
optional decimals indicated by one  ASCII ``.'' (full stop) character
(hexadecimal value 2e), or text, with arbitrary values except as stated
below, encoded in UTF-8.  

\item since the equals and linefeed characters
thus have special meaning, they are illegal in both labels and values. A
full-stop character may occur zero or once in a numerical value, zero
to N in a textual value.  

\item as a minimum, a shuttle contains a timestamp and a sequence number.

	\begin{itemize}
	\item a shuttle timestamp last label item is "timestamp", with the
	rest of the hierarchy being determined by the organisation. It's value
	is always a number, a high-precision variant of posix time: The whole
	part a decimal representation of the number of seconds elapsed since
	midnight on the 1 January 1970 (UTC), the decimal part representing
	sub-second precision, the first digit representing $\frac{1}{10}$s, the
	second $\frac{1}{100}$s, etc..

	\item a shuttle sequence number is a monotonic whole number that increases
	by one. The main sequence that should always be present is the sequence
	counter of the \cauldron{} component, which is system-wide, by \node{}:
	This sequence allows all consumers of any \rawproto{}-based protocol
	to discover lost shuttles instantly by comparing incoming shuttle
	sequence numbers to the previous shuttle's sequence: it should be exactly
	one higher. If the difference is more than one, this indicates shuttles 
	have been lost. While this is not an issue in message-based raw \rawproto{}
	clients, \diffproto{} clients may choose to mark such a connection as
	unreliable, or open a new connection to get a full cache dump, to make
	sure no information was missed.
	
	\end{itemize}

\item a value that is derived from one single other value, will append a 
label-item to the label of the value it was derived from.  

\end{itemize}

\begin{figure}[ht]
    \resizebox{1.0\hsize}{!}{\includegraphics{figures/protocolexample.eps}}
	\caption{Examples of \rawproto{} shuttles} \label{fig:shuttleexamples}
\end{figure}


\fig{shuttleexamples} shows 2 shuttles captured from our
experimental setup at client. The first shuttle is on lines 1 to 32, the
second, on lines 33 to 35. (the empty lines serving as shuttle separators
are considered as part of the shuttle).  The shuttle on lines 1 to 32
has a set of records, most of them numeric, and most of them related to
measures emanating from collectd agents deployed on the target hosts,
or contributed from a remote testing probe (line 27 is from such a
remote probe). Some, however, are internal measures contributed by the
cauldron or \rawproto{} output, as in lines 13 and 18, which give the
incrementing shuttle sequence number, and local timestamp of the central
host.  Some are derived values. Line 20 is a direct measure of free disk
space, in bytes. On line 21, a read/write \witch{} has consolidated this
value with the 2 related ones for this specific disk partition (free
and reserved), and has calculated that 2107334656 bytes corresponds to
50.6\% free space. On line 22 another read/write \witch{} has taken
that percentage value, compared it to some appropriate threshold and
decided that for this disk partition, 50.6\% free was too little,
but not yet cause for alarm. It contributed a state suffix to the
percentage measure indicating a state of ``1'' meaning "warning', and,
on line 23 appended a human-readable explanation to that warning. Note
that the namespace chosen at client site is hierarchical, according
to reverse-dns fashion as is usual, for example, in Java package names.
This allows, for example, for a service provider to gather data from many
organisations without name collisions.

The second shuttle on lines 33 to 35 has no real measures. It consists
entirely of the 2 internal measures sent by the dispatcher, to confirm
that the connection is live, and to allow continuous comparison between
the clocks of the dispatcher and the local clocks. We would recommend
that a shuttle be dispatched every tenth of a second, with or without
actual data.

\section{Transports}

Depending on the component it is used in, the \rawproto{} will be carried
on different transport protocols: It is designed to carry its own sequence
indicators, and maintains this even on message-based transports, to
facilitate development of clients, and to allow multiple shuttles per
message in message-based transports.

Two cases of \rawproto{} over message-based transports are inherent in
the backend side of the prototype design, in the local network context
of \node{}s. These are not used in the front-end strategy, and hence
never travel over any actual wires or routers.

Preliminary experiments used the UDP multicast facility present in the
operating system as a low-level publish-subscribe mechanism: \witches{}
wishing to contribute records could merely transmit shuttles in UDP
packets to a local multicast address. \witches{} wishing to obtain records
would join that multicast group, and receive copies of all shuttles
thus transmitted.

Although, theoretically, transmission errors and lost packets are
possible in this configuration, the design of the multicast code in
e.g. the Linux kernel makes this extremely unlikely. The worst-case in
practice: lost packets, is not a major issue in our design, as the 
message-based transports are only used in high-bandwidth, local,
short-circuited protocol stacks, such as the local interface (lo)
of Unix-like systems, and in that context we're retransmitting 
all measure results constantly at a high rate. A lost packet's
values will be resent within a very short period, all the time.

Any contributor in this design accumulates records into a local
buffer. A running calculation is maintained as to how large a shuttle the
accumulated data would represent. The records are sent as one shuttle,
as soon as either the next records would not fit into the maximum size
of UDP packet allowed, or one tenth of a second have passed since the
last such packet was sent.

With this technique, the \cauldron{} component could be considered
another case of COTS reuse: It merely consists of the local network of
the \node{}(s), and the multicast facilities of the local IP stack.

Later experiments have used the ZeroMQ framework to provide message-based
PUB-SUB and PUSH facilities. The strategy is the same as in the original
multicast facility: a contributor accumulates records until a configurable
maximum is reached or one tenth of a second have passed, after which the
resulting shuttle is encapsulated into a ZeroMQ message, and transmitted
from a 0MQ PUSH type socket towards a 0MQ PULL socket listening in a
cauldron process, whose only role is to transmit the shuttle over a 0MQ
PUB socket, from where it will be received by all 0MQ SUB type sockets
that have subscribed to it.

With this technique, the \cauldron{} component is a small daemon program,
to be developed.  The 0MQ technique is now preferred after concerns
were raised about the availability of effective multicast semantics in
cloud-based environments. We'll design our API in such a way as to be
agnostic of which technology the \cauldron{} component is based upon.

In both cases above, the data per message is exactly one shuttle,
as described, including the end-of shuttle marker, however, any code
should interpret messages received as containing 1--n shuttles: multiple
shuttles per message are likely to occur in later versions.

The input, output and input\slash{}output \witches{} that are managed by the
\coven{} component read lines of text from their stdin file descriptor,
write lines of text to their stdout file descriptor, or both. The \coven{}
component presents them with copies of shuttles containing values limited
to their subscription, and\slash{}or reads shuttles presented. The Unix
Pipe mechanism used in both directions is essentially a half-duplex
stream: shuttles are written back-to-back as they appear. The
\witch{} reads "lines of text" from stdin, accumulating or handling
records as they appear, one per line (see \fig{rawproto}).
An empty line signified a shuttle has just ended, and reading zero bytes
signifies that the \coven{} component wishes for the \witch{} to shut down.

\label{sec:rawproto}

``Raw'' \rawprotoacronym{} is essentially shuttles (see
\fig{rawproto}) of all measures, within one tenth of a second
of their being measures, sent back-to-back over an HTTP connection,
in what amounts to an infinitely long ``file'' whose size is never
known beforehand.  The approach is a pragmatic way to leverage the
ubiquity of the HTTP protocol, its wide availability and accessibility
in many environments. In practice, raw \rawproto{} will occur rarely,
as the bandwidth requirements may be prohibitive for general Internet
use. Outside the \node{} environment, \diffproto{} will be preferred.

\diffproto{} is exactly the same as \rawproto{} (see \ref{sec:rawproto}),
with the following exceptions:

\begin{itemize}

\item The first shuttle(s) sent after the HTTP response header consist
of a dump of all values that the sender knows of (and that match
the URL requested) at that moment in time. These special shuttles
contain an indicator record of ``cachedump'' with value ``true''
within the appropriate namespace. Although it is expected that in most
implementations this ``key frame'' will be sent in one (potentially
huge) shuttle, implementations should not count on this and use the
``cachedump'' indicator instead, if they need to distinguish between
these and subsequent ``differential'' shuttles.

\item The first shuttle sent that has no cachedump indicator, and all
subsequent shuttles of the current HTTP connection are ``differential''
shuttles: They do not repeat state already sent, but only indicate changes
to previous state.  Because many values change rarely, even when measured
frequently, this allows for huge bandwidth gains.

\end{itemize}

A \diffproto{} client that needs to compare previous or related values
with incoming updates may wish to keep a local cache that is instantly
provisioned after the cachedump shuttle(s) and kept up-to-date by
differential ones.  In such clients, \diffproto{} effectively is a remote
cache update protocol, and it thus becomes easy to create \diffproto{}
proxies to limit bandwidth usage, for example, if multiple clients on
a LAN require the same, or subsets of the same namespace.

\diffproto{} clients that merely need to respond to a particular subset of
the namespace, without requiring state, may safely ignore the differential
nature of the protocol: The cache dump will, in their case merely serve
to supply an initial state.

In practice, The \diffproto{} protocol, as such is expected to occur
mostly on LAN's, or, on public services. Most organisations will probably
chose to use one of the secured variants, known as \secproto{}

\label{sec:compproto}

Even casual visual inspection of a few \rawproto{} shuttles immediately
highlights one striking characteristic of \rawproto{} labels: redundancy.
All labels carry the same organisation prefix, and since one host in one
system in one realm can have thousands of measures being streamed in, a
large part of those longer prefixes will be repeated, many times, as well.

\begin{figure}[ht]
	\centering
    \resizebox{1.0\hsize}{!}{\includegraphics{figures/small_shuttle.eps}}
    \caption{Small \rawproto{} shuttle before compression}
	\label{fig:smallshuttle}
\end{figure}

\fig{smallshuttle} shows a small shuttle of 6 records, with
much redundancy in the labels: All the labels are identical up to and
including the 6th label part \emph{df}, and even then half the records
have \emph{boot.free.percentage} following that one common label prefix.
The prefixtree encoding attempts to remove some of that redundancy by

\begin{itemize}

\item expressing the shuttle as a tree, with nodes created according
to label parts, from left to right, containing either more nodes
and\slash{}or values. \fig{smallshuttletree} shows our small
example, thus represented, and with the tree levels below.

\begin{figure}[ht]
    \centering
     \resizebox{1.0\hsize}{!}{\includegraphics{figures/small_shuttle_tree.eps}}
    \caption{Small \rawordiffproto{} shuttle represented as a tree}
    \label{fig:smallshuttletree}
\end{figure}

\item collapsing tree nodes with only 1 child that is not a value to a
single node with the combined names of the collapsed nodes. \fig{smallshuttlecollapsedtree}
shows out small example tree, in this collapsed state: Only 4 tree levels remain.

\begin{figure}[ht]
    \centering
     \resizebox{1.0\hsize}{!}{\includegraphics{figures/small_shuttle_collapsed_tree.eps}}
    \caption{Small \rawordiffproto{} shuttle represented as a collapsed tree}
    \label{fig:smallshuttlecollapsedtree}
\end{figure}

\item walking the resulting tree outputting lines containing the tree
depth, the node name, and optionally, an equals sign and the node value
as more formally described in \fig{prefixtree}).

\end{itemize}

\begin{figure}[ht]
	\centering
    \includegraphics{figures/prefixtree.eps} \caption{The
   \compproto{} Shuttle} \label{fig:prefixtree}
\end{figure}

\fig{smallshuttleencoded} shows the result of prefixtree-encoding
the small shuttle from \fig{smallshuttle}: In line 1, the
collapsed nodes \emph{\clientc,\clientcn,\clientou,\clientrealm,\clientsys, and df} from their
corresponding label parts in lines 1--6 of \fig{smallshuttle}
occur only once, at tree level zero, thereby indicating that all lower
levels should be prefixed with this. In lines 2 and 5, the nodes at level
one in the tree, express the 2 further label part possibilities, either
\emph{home.free.percentage} or \emph{boot.free.percentage}.  Note that
in these cases, these node have their own values as well as expressing
prefixes for further levels. This is a consequence of the \rawproto{}
convention that derived values suffix the labels of their source, and
indicates \emph{percentage.state} was derived from \emph{percentage}. The
mechanism continues in the same manner into levels 2 and 3 of the tree,
on lines 3,4,6 and 7 of \fig{smallshuttleencoded}

\begin{figure}[ht]
	\centering
    \includegraphics{figures/small_shuttle_prefixtree_encoded.eps}
    \caption{Small \rawordiffproto{} shuttle
    encoded as a \compproto{} Shuttle} \label{fig:smallshuttleencoded}
\end{figure}

With the kind of hierarchy shown, prefixtree-encoding easily reduces
the size of shuttles by 50\% or more for trivially small shuttles,
individually encoded. We expect far better ratios for larger shuttles,
and for variants of the protocol that maintain state between shuttles.

\secproto{} and \seccompproto{} are \diffproto{} or \compproto{}
transported over HTTPS instead of HTTP.  The acronyms therefore covers
a range of implementations, SSL and TLS variants, authentication
schemes, key sizes, etc..  Its exact technical implementation at
any configuration will depend on the requirements, and rules of the
organisation involved. All the back-end components implement only the
plaintext variants of the protocols, and make no attempt at authentication
or authorization whatsoever, and it is left to the front-end webserver
and remote clients to agree on the exact security requirements per case.
The only requirement for an implementation to qualify as being \secproto{}
resp. \seccompproto{} is that after the authentication was successfully
performed, the payload transported (encrypted) by the resulting
protocol is exactly \diffproto{} or \compproto{}. We expect most real-life
traffic over the Internet will be \seccompproto{}, compressed differential
\rawproto{} encapsulated inside a HTTPS (TLS) stream.


\newcommand*\svgtag[1]{\textless{}#1\textgreater{}}

\chapter{SVG Viewer Namespace} 

\setlength\fboxsep{4mm}
\fbox{%
\begin{minipage}{.9\linewidth}
\emph{First of all, a word of warning: ExtreMon Display is subject to intense
discussion and no small measure of thinking on my part into where to
take it, technology-wise. Drawing the "animatible" SVG schematic with
all details is no trivial task, and I'm thinking on how to generate it
rather than requiring someone to draw it. Also, the underlying Apache
Batik library gives smooth change, zoom, rotate, all I could wish for,
\emph{but} requires a J2EE runtime environment, and I have serious doubts
about the future of Java on the workstation. A JavaScript solution is being
pondered and experimented with, but so far the major stumbling block
is the rendering performance (or lack thereof) of SVG on browsers: zooming
non-trivial SVG takes seconds in the best case. This, and the lack of
layoutmanagers in SVG leads me to consider everything between JavaScript
Canvas and OpenGL even including (ab)using a 3D game engine. This means
the information below may change, dramatically, unless the SVG engines
inside the major browsers get much, much better, soon.}
\end{minipage}}

\vspace{1cm}

\begin{figure}[!ht]
	\resizebox{.9\textwidth}{!}{
	\includegraphics{figures/svgexample.eps}} \caption{An X3Mon SVG Namespace Example}
	\label{fig:svgexample}
\end{figure}

To effectively map \rawproto{} records onto SVG elements in an in-memory
representation of an SVG Document, several new attributes are introduced
to mesh with existing SVG elements:

\section{SVG Namespace}

\subsection*{The x3mon:define attribute}

\psframebox{$\text{\textless{}svg-tag \textcolor{\hilitecolor}{x3mon:define="variable name:[attr name[/attr-part-name]]"} svg-attr="value" [\ldots{}] \slash{}\textgreater{}}$} 

The x3mon:define attribute creates a named variable referring to the
value of an attribute or the text content of the SVG element that
contains it. This allows for arbitrary attributes of any SVG elements
in the same document to be used in subsequent mappings, so that these
can be managed visually from within e.g. an SVG editor. For example, I
keep rectangles outside the document boundaries, that define the colors
representing state (ok, alert, warning, ..), so these become easy to
change from within the document. Lines 3--5 of \fig{svgexample} show
define attributes being used to establish the variables ``statestyle[0]''
to ``statestyle[2]'', containing the style information of the elements
they occur in. At runtime, this will result in ``statestyle[0]''
having the value ``fill:\#79ffb3'', for example. Note that although
the notation suggests some kind of subscript mechanism, this is only
implied, the square brackets are simply part of the name here. The
optional syntax attribute name\slash{}attribute part name allows to set
the resulting variable to part of the attribute's value. Currently, the
only attribute on which this is defined are CSS svg:style attributes,
where the resulting variable may be defined to one part of the element's
CSS style, its primary use being to use that part in a later x3mon:map
sset action (see below)

\subsection*{The x3mon:usage attribute}

\psframebox{$\text{\textless{}g \textcolor{\hilitecolor}{x3mon:usage="template" svg:id="template-id"}\textgreater{} [svg-elements\ldots{}] \textless{}g\slash{}\textgreater{}}$}

The x3mon:usage attribute marks an SVG element as being suitable to
be used in a specific role. Currently, only the value "template" is
valid which indicates that the element is an SVG group suitable to be
used as an Extremon template. Lines 7--12 of \fig{svgexample}
form an x3mon template because of this use of x3mon:usage in its
topmost SVG <g> element. The identity of this template is determined
by its SVG id attribute.  This is the only instance where the SVG
id is significant for our purposes (it is ignored everywhere else).
This exception was made so as to be able to make
use of the SVG templating mechanism, the SVG <use> tag, as seen in
lines 18 and 21 of \fig{svgexample}, where our template used,
twice. Re-using the existing SVG mechanism allows standard SVG editors
to represent our template usage, which greatly facilitates editing.
The similarity ends there, though, since x3mon will replace <use>
directives by \emph{copies} of the template to allow treating them
differently to one another, where the original <use> elements all remain
identical to the template. \fig{svgsubst} shows the same SVG
document after template instantiation.

\begin{figure}[!ht]
	\resizebox{.9\textwidth}{!}{
	\includegraphics{figures/svgsubst.eps}} \caption{An X3Mon SVG Namespace with templates replaced} \label{fig:svgsubst}
\end{figure}

In \ref{fig:svgsubst}, the SVG <use> elements of \fig{svgexample}
have both been replaced by the SVG <g> they referred to. The <g> that
was marked as the template has been removed. Note that this substitution
will only occur with <use> elements that refer to <g> elements marked
as x3mon templates: Any other <use> elements will be left undisturbed.

\subsection*{The x3mon:id attribute}

\psframebox{$\text{\textless{}svg-tag \textcolor{\hilitecolor}{x3mon:id="label part[.label part]\ldots{}"} [svg-attr="value"]\ldots{} \slash{}\textgreater{}}$}

The x3mon:id attribute suffixes the \rawproto{} namespace of the
containing SVG elements, making the identity of the element that contains
it and those contained by it more specific.  The \rawproto{} identity
of any SVG element in the document is the concatenation, in depth-first
traversal order, of any x3mon:id attributes encountered.  Practically,
this means that SVG group elements may semantically group all the
child elements they contain into a single branch of the \rawproto{}
namespace, requiring its children to have only more detailed label
parts. This establishes a flexible hierarchy of identities inside the SVG
DOM. \fig{svgsubst} shows the element's resulting identities,
in gray.

\subsection*{The x3mon:map attribute}

\psframebox{\resizebox{.9\textwidth}{!}{%
\textless{}svg-tag \textcolor{\hilitecolor}{x3mon:map="$%
\left[\text{label suffix}\right]\text{:}\left\{%
\begin{tabular}{c} 
nset \\ 
tset \\ 
tsset \\ 
cdset \\ 
sset \\ 
\begin{sideways}\tiny{\ldots{}}\end{sideways}
\end{tabular} \right\}\text{([attr]:\#\textpipe{}format[:modifier])}$"} svg-attr\ldots{} \slash{}\textgreater{}}}

The x3mon:map attribute is a list of actions (semicolon-separated, as
inspired by the SVG style attribute), each one creating a relationships
between the \rawproto{} namespace (or sub-namespace) and an attribute
or the text content of the SVG element that contains it

Lines 12,13,18, and 19 of \fig{svgsubst} show map attributes.
Lines 12 and 18 have two, while lines 13 and 19 only specify one each.
The nset actions show here don't specify a namespace suffix (the
field is empty, we start with the colon used as separator, meaning
they apply to the x3mon:id of the element that contains them. The tset
actions do specify a suffix meaning they apply to the x3mon:id of their
elements with that suffixed.  In plain English, lines 12 and 13 of
\fig{svgsubst} signify:

\emph{When a record with id
\clientprefix.app1.cpu.0.cpu.steal.value arrives, set
the width of the \svgtag{rect} element in line 12 to its value, and the
text content of the \svgtag{text} element from line 13 to ``(its value
truncated without decimals) \% steal''}

\emph{When a record with id
\clientprefix.app1.cpu.0.cpu.steal.value.state
arrives, construct a variable substituting ``\#'' for its value in
``statestyle[\#]'', look up its value, and set the style attribute of
the \svgtag{rect} element from line 12 to that value.}

Some essential action types are defined below (but this syntax was
designed to be extensible); all these set the attribute or text content
of the element that contains them:

\begin{itemize} 
\item \emph{nset} (set numeric) formats a value using printf-like syntax, as a floating-point number
\item \emph{tset} (set text) formats a value using printf-like syntax, as a string of text
\item \emph{tsset} (set timestamp) formats a value given in seconds since the
Unix-era into a timestamp
\item \emph{cdset} (set countdown) formats a value given in seconds to the period of time those seconds represent in
years, months, days, weeks, days, hours, minutes and seconds.  (this is an approximation since leap years (nor seconds) are taken into account).
\item \emph{sset} (set style) formats a value using printf-like syntax, and set an individual style element within the target's
svg:style attribute to the resulting formatted string. This allows for SVG using CSS style (as opposed to distinct style attributes),
to still have individual style elements impacted by incoming data. The ``attribute'' argument is used to determine the style
element, as the attribute is hard-coded to ``style''.
\end{itemize}

The original SVG document, after defines have been recorded, templates
replaced, and identities established, is ready to respond to some
\rawproto{} records.
